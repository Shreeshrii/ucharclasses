\documentclass{article}
\usepackage{polyglossia,fontspec,xcolor,metalogo}
% To allow copying of Devanagari from pdf
\XeTeXgenerateactualtext=1
% englishfont, devanagarifont, vedafont
\setmainfont{FreeSerif}
\newcommand{\lmr}{\fontfamily{lmr}\selectfont} % Latin Modern Roman
\defaultfontfeatures{Mapping=tex-text}
\setmainlanguage{english}
\setotherlanguage{sanskrit}
\newfontfamily{\englishfont}[Scale=1.1]{FreeSerif}
\newfontfamily{\devanagarifont}[Script=Devanagari,Scale=1]{Siddhanta}
\newfontfamily{\vedafont}[Scale=1.2,Color=red]{Siddhanta}
% Modified ucharclasses/sty
\usepackage[Latin, GeneralPunctuation, DevanagariMarks, DevanagariPreMarks, DevanagariExtended, DevanagariPostMarks, VedicExtensions, SuperscriptsAndSubscripts]{ucharclasses}
\setTransitionsForLatin{\englishfont}{}
\setTransitionTo{GeneralPunctuation}{\englishfont}
\setTransitionTo{DevanagariPreMarks}{\devanagarifont}
\setTransitionTo{DevanagariPostMarks}{\devanagarifont}
\setTransitionTo{DevanagariMarks}{\vedafont}
\setTransitionTo{DevanagariExtended}{\vedafont}
\setTransitionTo{VedicExtensions}{\vedafont}
%
\setTransitionsFor{Latin}
  {\hyphenrules{english}\englishfont}
  {\hyphenrules{sanskrit}\devanagarifont}
\setTransitionsFor{Devanagari}
  {\hyphenrules{sanskrit}\devanagarifont}
  {\hyphenrules{english}\englishfont}
%
\setlength{\parindent}{0pt}
%
\begin{document}
\obeylines 
\section*{Introduction}
This trial is to use \emph{ucharclasses.sty} to easily switch between Devanagari and Latin scripts without additional {\fontfamily{lmr}\selectfont \LaTeX} directives.
\medskip
यह परीक्षण लेटेक निर्देशों के बिना, आसानी से, हिन्दी और अंग्रेजी के बीच स्विच करने के लिए ucharclasses का उपयोग करता है। 

\section*{Vedic Sanskrit in Devanagari Script}
This document shows how to use the combining marks and characters from Devanagari, DevanagariExtended and VedicExtensions Unicode blocks to display the Vedic marks in red color.

\medskip\hrule\medskip

ओमिति॒ ब्रह्म॑ । ओमिती॒दꣳसर्वम्᳚ ।

ॐ
गं॒ध॒द्वा॒रां दु॑राध॒र्षां॒ नि॒त्यपु॑ष्टां करी॒षिणी᳚म् । 
ई॒श्वरी॑ꣳ सर्व॑भूता॒नां॒ तामि॒होप॑ह्वये॒ श्रियम् ॥ ९॥ 

\medskip\hrule\medskip
\section*{Technical Info}

This was typeset using \XeLaTeX \ \the\XeTeXversion\XeTeXrevision . A modified version of the package \emph{ucharclasses.sty} was used for automatic font switching based on Uniocde ranges.
\end{document}
